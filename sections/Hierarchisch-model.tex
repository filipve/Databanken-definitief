\newpage

\section{Hiërarchisch model  }

Zie voor meer info in slides 09 \_ 2 \_ Hierarchisch (2).pdf

\subsection{Wat zin de bouwstenen van een hiërarchisch model?}

\begin{itemize}
\item Segmenttypes
\item Parent-child relationship-types
\item Wortelsegment – bladeren
\item n-m verbanden zijn niet toegelaten
\end{itemize}

\subsection{wat zijn de talen van de gegevensbanksystemen?}
\begin{itemize}
\item Gegevensdefinitietaal : DDL
\item Gegevensmanipulatietaal : DML
\end{itemize}


\subsection{Wat zijn de functiecodes van de gegevensmanipulatietaal : DML?}

GU = overloopt boom en stopt bij eerst gevonden segment\\\\

\begin{theo}[Voorbeeld GU]{thm:GU}

GU DOKTER(DFNAAM = DEPRET)

       PATIENT

GU DOKTER(DFNAAM = DEPRET)

       PATIENT(PFNAAM = MAES)
       
       CONSULT(CDATUM = 16092005)

\end{theo}




\noindent GN : zoekt verder en stopt bij eerst gevonden segment

Voorbeeld:

GN PATIENT

GN PATIENT(PFNAAM = MAES)


\noindent GNP : zoekt verder maar alleen in de afhankelijke segmenten van een parent

Voorbeeld:

GU DOKTER (DNAAM = DEPRET)

IF STATUS\_CODE = SPACE

           DISPLAY I\_O\_DOKTER
           
           GNP PATIENT
           
           PERFORM UNTIL STATUS\_CODE NOT = SPACE
           
                   DISPLAY I\_O\_PATIENT
                   
                   GNP PATIENT
                   
            END-PERFORM
END-IF

\noindent GHU - GHN - GHNP : analoog maar H moet voor een delete of een replace

Voorbeeld:

GHU DOKTER (DNR = 12)

DLET

\subsection{Wat is het verschil tussen GN en GNP?}