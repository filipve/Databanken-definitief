\newpage

\section{Indexen}

Zie pdf 07 \_ 1 \_ INDEXen.pdf

\subsection{Wat is het doel van een index?}
Beïnvloeden van de verwerkingstijd.
\subsection{Wat zijn indexen? }

\subsection{Wat is het verschil met een clustered index?}

\subsection{Wat is een index? }
Voordeel: index versnelt verwerking
Nadeel:
\begin{itemize}
    \item Index neemt opslagruimte
    \item Elke mutatie vraagt aanpassing van index => verwerking vertraagt
\end{itemize}
	
\subsection{Wanneer gebruik en waarom? }

\subsection{Geef een voorbeeld m.b.v de bijgevoegde SQL-tabellen}

\subsection{Wat zijn de methodes voor het opzoeken?}

\begin{itemize}
\item Sequentiële zoekmethode : rij voor rij
    \begin{itemize}
        \item Tijdrovend en inefficiënt
    \end{itemize}
\item Geïndexeerde zoekmethode : index (B-tree)
    \begin{itemize}
        \item Boom
        \item knooppunten
        \item Leafpage
        \item 2 methodes
            \begin{itemize}
            \item Zoeken van rijen met een bepaalde waarde
            \item	Doorlopen van de hele tabel via een gesorteerde kolom (geclusterde index)
             \end{itemize}
    \end{itemize}
\end{itemize}


\subsection{Welke speciale vormen van indexen bestaan er?}

Multi-tabelindex = index op kolommen in meerdere tabellen\\\\
Virtuele-kolomindex = index op een expressie\\\\
Selectieve index = index op een gedeelte van de rijen\\\\
Hash-index = index op basis van het adres op de pagina\\\\
Bitmapindex = interessant als er veel dubbele waardes zijn